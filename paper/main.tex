\documentclass[a4paper, oneside, openany]{book}
\usepackage{graphicx}
\graphicspath{{./images/plots/}}
\usepackage{subfiles}
\usepackage[italian]{babel}
\usepackage{hyperref}
\usepackage[export]{adjustbox} 
\usepackage{amsmath}
\usepackage{csquotes}
\usepackage{changepage}
\usepackage{mathtools}
\usepackage[table]{xcolor}
\usepackage{booktabs}
\usepackage{colortbl}
\usepackage{array}
\usepackage{float}
\usepackage{multirow}
\usepackage{makecell}
\usepackage{adjustbox}

\usepackage{titlesec} % Modifica il titolo del capitolo
\titleformat{\chapter}[hang]{\huge\bfseries}{\thechapter}{1em}{}

% Per generare la bibliografia
\usepackage[
backend=biber,
style=ieee,
sorting=none 
%sorting=ynt
]{biblatex}
\addbibresource{files/bibliografia.bib}

\definecolor{headerblue}{RGB}{0, 102, 204}
\definecolor{rowgray}{gray}{0.95}

\begin{document}
 
    % File di frontespizio
    \subfile{files/frontespizio.tex}

    % Numerazione delle prima pagina con i numeri romani
    \pagenumbering{roman}
    
    \tableofcontents % Indice
    \listoffigures % Indice delle immagini
    \listoftables % Indice delle tabelle

    \clearpage

    % Numera le pagine con i numeri arabi e resetta il counter da questa pagina in poi
    \pagenumbering{arabic} 
    \setcounter{page}{1}

    \subfile{files/body/0_introduzione.tex}
    \subfile{files/body/1_stato_arte.tex}
    \subfile{files/body/2_modello.tex}
    \subfile{files/body/3_implementazione.tex}
    \subfile{files/body/4_risultati.tex}
    \subfile{files/body/5_conclusioni.tex}

    \appendix
    \subfile{files/appendice.tex}

    \printbibliography

\end{document}
