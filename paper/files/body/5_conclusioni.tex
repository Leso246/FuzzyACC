\chapter{Conclusioni}
\label{cha:conclusioni}

\noindent Il lavoro presentato ha avuto come obiettivo la progettazione e l'implementazione 
di un sistema di \textit{Adaptive Cruise Control} (ACC) basato su logica fuzzy.  
Attraverso la definizione di variabili linguistiche, delle relative membership functions 
e di un insieme di regole, è stato possibile modellare un comportamento 
di guida realistico.
\\\\
\noindent I risultati ottenuti mostrano che il modello simulato riesce a riprodurre con buona 
fedeltà le dinamiche di un sistema ACC reale.  
In particolare:
\begin{itemize}
    \item il confronto tra le accelerazioni simulata e reale evidenzia una forte correlazione, 
    confermando la capacità del modello di catturare le variazioni principali, con una risposta più smussata;
    \item la velocità simulata si è dimostrata altamente coerente con quella reale e con quella del veicolo leader, 
    garantendo una guida fluida e priva di oscillazioni indesiderate;
    \item lo space gap mantenuto dal modello è risultato più prudente rispetto a quello osservato nei dati reali, 
    ma meno conservativo delle prescrizioni più restrittive (ad esempio quelle dell'ACI), 
    suggerendo un compromesso tra realismo e sicurezza;
    \item la simulazione in condizioni meteorologiche avverse ha evidenziato un comportamento più cauto, 
    con distanze di sicurezza maggiori e frenate più marcate nella fase iniziale, confermando 
    la corretta influenza della variabile \texttt{weather\_condition}.
\end{itemize}

\noindent Nel complesso, il sistema proposto ha dimostrato di essere in grado di coniugare sicurezza, comfort e realismo, 
riproducendo strategie di guida comparabili a quelle adottate da un ACC commerciale.  
La logica fuzzy si è rivelata uno strumento efficace per gestire l'incertezza e la variabilità tipiche del contesto autostradale, 
garantendo decisioni graduali e interpretabili.
\\\\
\noindent Le limitazioni principali del lavoro sono legate all'assenza di una validazione su veicolo reale 
e alla semplificazione adottata nell'ipotesi che l'accelerazione impartita coincida con quella effettiva.  
Nonostante ciò, i risultati ottenuti offrono una base solida per sviluppi futuri, 
orientati verso una maggiore accuratezza e un'integrazione con approcci ibridi che combinino fuzzy logic, 
machine learning e controllori predittivi.

