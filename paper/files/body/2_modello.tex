\chapter{Modello Proposto}

Il presente capitolo descrive la progettazione del sistema fuzzy impiegato per l'Adaptive Cruise Control.  
\\\\
Si considera un veicolo di categoria M1 (auto destinata al trasporto di persone, con al massimo otto posti a sedere oltre al conducente)
dalle prestazioni medie in termini di accelerazione e decelerazione.  
\\\\
L'ambiente di riferimento è quello autostradale, in cui i veicoli si muovono in traiettorie rettilinee.  
\\\\
D'ora in avanti, il veicolo che segue verrà indicato come \emph{ego} (ossia il veicolo dell'utilizzatore del sistema ACC), 
mentre il veicolo che precede verrà indicato come \emph{leader}.  
\\\\
Per le velocità di esercizio si è scelto un intervallo compreso tra \(70\,\text{$\frac{\mathrm{km}}{\mathrm{h}}$}\) e \(150\,\text{$\frac{\mathrm{km}}{\mathrm{h}}$}\): 
il valore minimo riflette le condizioni tipiche di marcia autostradale, mentre il valore massimo coincide con il limite consentito 
sulle autostrade italiane (generalmente \(130\,\text{$\frac{\mathrm{km}}{\mathrm{h}}$}\), elevabile a \(150\,\text{$\frac{\mathrm{km}}{\mathrm{h}}$}\) in circostanze particolari~\cite{limite_autostrada_150}).  
 

% Descrive le variabili linguistiche e mf
\subfile{2_modello/variabili_linguistiche_e_mf}

\section{Creazione delle Regole}
Per la definizione delle regole di controllo sono state considerate tutte le possibili combinazioni dei termini linguistici 
delle variabili in input, in modo da garantire la copertura di tutti i possibili scenari.  
Il numero totale delle regole si ottiene moltiplicando il numero di termini di ciascuna variabile di input:
\[
n^\circ \text{ regole} = 2 \times 5 \times 5 = 50
\]

\noindent Tale numero è relativamente contenuto per un controllore fuzzy, anche grazie agli accorgimenti già illustrati nella
Sezione \ref{subsubsection:th_motivazione}; per questo motivo si è deciso di non adottare un 
approccio a cascata, privilegiando invece una singola base di regole che mantiene la struttura del sistema più semplice.  
\\\\
\noindent Una volta generate tutte le combinazioni, è stato quindi assegnato il termine linguistico di output ritenuto più appropriato 
(relativo alla variabile \texttt{acceleration}).    
L'elenco completo delle regole è riportato in Appendice~\ref{tab:regole_fuzzy}.  
\\\\
\noindent Si evidenzia che, a parità di valore delle altre variabili, in condizioni meteorologiche peggiori sono stati scelti 
output di accelerazione più prudenti, al fine di riflettere una maggiore attenzione alla sicurezza.