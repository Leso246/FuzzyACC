\chapter{Modello Proposto}

Il presente capitolo descrive la progettazione del sistema fuzzy impiegato per l'Adaptive Cruise Control.  
\\\\
Si considera un veicolo di categoria M1 (auto destinata al trasporto di persone, con al massimo otto posti a sedere oltre al conducente)
dalle prestazioni medie in termini di accelerazione e decelerazione.  
\\\\
L'ambiente di riferimento è quello autostradale, in cui i veicoli si muovono in traiettorie rettilinee.  
\\\\
L'auto che segue verrà indicata come \emph{ego}, mentre quella che precede come \emph{leader}.  
\\\\
Per le velocità di esercizio si è scelto un intervallo compreso tra \(70\text{ km/h}\) e \(150\text{ km/h}\): il valore minimo riflette le condizioni tipiche di marcia autostradale, il valore massimo coincide con il limite consentito sulle autostrade italiane (generalmente \(130\text{ km/h}\), elevabile a \(150\text{ km/h}\) in circostanze particolari~\cite{limite_autostrada_150}).  

% Descirve le variabili linguistiche modellate
\subfile{2_modello/variabili_linguistiche}

\section{Definizione delle Membership Functions}
\subsection{Input}
\subsection{Output}

\section{Creazione delle Regole}