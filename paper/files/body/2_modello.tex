\chapter{Modello Proposto}

Il presente capitolo descrive la progettazione del sistema fuzzy impiegato per l'Adaptive Cruise Control.  
\\\\
Si considera un veicolo di categoria M1 (auto destinata al trasporto di persone, con al massimo otto posti a sedere oltre al conducente)
dalle prestazioni medie in termini di accelerazione e decelerazione.  
\\\\
L'ambiente di riferimento è quello autostradale, in cui i veicoli si muovono in traiettorie rettilinee.  
\\\\
D'ora in avanti, il veicolo che segue verrà indicato come \emph{ego} (ossia il veicolo dell'utilizzatore del sistema ACC), 
mentre il veicolo che precede verrà indicato come \emph{leader}.  
\\\\
Per le velocità di esercizio si è scelto un intervallo compreso tra \(70\,\text{$\frac{\mathrm{km}}{\mathrm{h}}$}\) e \(150\,\text{$\frac{\mathrm{km}}{\mathrm{h}}$}\): 
il valore minimo riflette le condizioni tipiche di marcia autostradale, mentre il valore massimo coincide con il limite consentito 
sulle autostrade italiane (generalmente \(130\,\text{$\frac{\mathrm{km}}{\mathrm{h}}$}\), elevabile a \(150\,\text{$\frac{\mathrm{km}}{\mathrm{h}}$}\) in circostanze particolari~\cite{limite_autostrada_150}).  
 

% Descrive le variabili linguistiche e mf
\subfile{2_modello/variabili_linguistiche_e_mf}

\section{Creazione delle Regole}