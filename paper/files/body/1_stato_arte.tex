\chapter{Stato dell'Arte}
La crescente complessità del traffico stradale e la ricerca di una maggiore sicurezza hanno spinto l'evoluzione dei sistemi di 
assistenza alla guida (ADAS), tra cui l'Adaptive Cruise Control (ACC). L'ACC è progettato per mantenere una distanza di sicurezza 
costante dal veicolo che precede, gestendo automaticamente la velocità e la distanza tra i veicoli.
\\\\
\noindent Sebbene i controller PID siano ampiamente utilizzati per la loro robustezza e i bassi requisiti hardware, spesso necessitano di 
una calibrazione complessa per adattarsi ai diversi scenari di guida e non riescono a ottenere un controllo ottimale. Per superare 
queste limitazioni, la ricerca si è orientata verso approcci più flessibili come la \textbf{logica fuzzy} 
e il \textbf{Model Predictive Control (MPC)} \cite{singh2015adaptive}.

\subsection*{Applicazione della Logica Fuzzy nell'ACC}
I sistemi basati su logica fuzzy sono particolarmente adatti per gestire la complessità e l'incertezza del 
comportamento di guida. Essi modellano il processo decisionale umano, utilizzando variabili linguistiche e regole 
basate sull'esperienza del guidatore. La logica fuzzy permette di ottenere una risposta più fluida e naturale rispetto 
ai metodi di controllo tradizionali, migliorando il comfort di guida \cite{simic2022cascaded}.
\\\\
\noindent \textbf{Sistemi a Logica Fuzzy Gerarchici e a Cascata}: Per gestire un gran numero di variabili di input 
senza aumentare esponenzialmente il numero delle regole, sono stati proposti sistemi a cascata o gerarchici. 
In questo approccio più sistemi di inferenza fuzzy sono collegati tra loro, riducendo significativamente il numero 
complessivo di regole e il carico computazionale.In tal modo il sistema diventa adatto per applicazioni in tempo reale, 
pur mantenendo la capacità di gestire scenari di guida complessi \cite{simic2022cascaded}.
\\\\
\noindent\textbf{Confronto con altri approcci}: In un'implementazione su un'auto modello autonoma, è stato dimostrato che un controller 
fuzzy può fornire una risposta più fluida e mantenere un errore di distanza inferiore rispetto a un controller PID 
tradizionale \cite{alomari2020fuzzy}. Un'altra ricerca ha evidenziato come un ACC che combina una rete neurale e un algoritmo 
fuzzy superi le prestazioni delle reti neurali convenzionali.

\subsection*{Integrazione con altre tecnologie}
Studi più recenti hanno esplorato l'integrazione della logica fuzzy con altre metodologie di controllo avanzate. Ad esempio, 
una strategia di controllo gerarchica combina un osservatore dello stato del veicolo basato su machine learning, 
un \textbf{Fuzzy Model Predictive Controller (fuzzy-MPC)} e un controller esecutivo PID per migliorare la precisione di 
tracciamento e la stabilità del sistema. Questo approccio ibrido mira a bilanciare le prestazioni di tracciamento, il 
comfort e la robustezza del controllo in ambienti incerti \cite{guo2023adaptive}.
\\\\
\noindent In sintesi, i controller a logica fuzzy rappresentano una soluzione robusta e intuitiva per la progettazione di sistemi ACC, 
offrendo prestazioni superiori in termini di fluidità e comfort di guida rispetto ai controller classici. L'integrazione di 
questi sistemi con altre tecnologie, come l'MPC e il machine learning, apre la strada a soluzioni ancora più sofisticate, in 
grado di adattarsi dinamicamente a scenari di guida complessi e variabili.