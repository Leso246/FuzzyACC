\chapter*{Introduzione}  
\addcontentsline{toc}{chapter}{Introduzione} 
Negli ultimi anni, i sistemi avanzati di assistenza alla guida (ADAS, acronimo di Advanced Driver Assistance Systems) hanno assunto 
un ruolo centrale nella progettazione dei veicoli moderni, contribuendo a migliorare la sicurezza e il comfort del conducente. 
Tra questi sistemi, l'Adaptive Cruise Control (ACC) permette di mantenere automaticamente una distanza di sicurezza dal 
veicolo che precede, adattando la velocità del veicolo controllato in funzione delle condizioni del traffico.
\\\\
\noindent Il presente lavoro si concentra sulla progettazione di un sistema fuzzy per l'ACC, che sfrutta logiche di tipo 
linguistico per gestire in maniera graduale e naturale le accelerazioni e le decelerazioni del veicolo. L'approccio 
fuzzy si rivela particolarmente efficace per modellare comportamenti come quelli della guida in autostrada,
dove le velocità e le distanze tra veicoli variano continuamente.
\\\\
\noindent Nel paper viene illustrato come è stato progettato e implementato un sistema fuzzy, descrivendo la scelta 
delle variabili linguistiche, la costruzione degli insiemi fuzzy e la definizione delle regole di controllo. Inoltre, 
viene presentata una simulazione basata su dati reali per analizzare i risultati e valutare l'efficacia del sistema nel 
regolare la velocità e mantenere la sicurezza e il comfort durante la guida.



