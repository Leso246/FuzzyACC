\chapter{Lavori Futuri}

\noindent Nonostante i risultati promettenti, il lavoro svolto rappresenta un punto di partenza 
più che una soluzione definitiva. Vi sono infatti diversi aspetti che meritano ulteriori 
approfondimenti e sviluppi.
\\\\
\noindent Un primo miglioramento riguarda la gestione delle condizioni meteorologiche avverse.  
Sebbene il sistema tenda ad aumentare lo \texttt{space\_gap} in caso di \texttt{bad\_weather}, 
sono emerse frenate eccessivamente brusche, che potrebbero compromettere la sicurezza 
sul bagnato invece di incrementarla.  
Un'estensione del modello dovrebbe quindi includere strategie di decelerazione più graduali, 
tenendo conto della ridotta aderenza del manto stradale.
\\\\
\noindent Un altro sviluppo naturale sarebbe la validazione su veicoli reali, sia in scala (ad esempio con modelli radiocomandati), 
sia su veicoli veri dotati di sensori, per verificare il comportamento del sistema in scenari concreti.  
La sperimentazione pratica consentirebbe di affinare il modello e di confrontarne i risultati con situazioni dinamiche reali.
\\\\
\noindent Dal punto di vista funzionale, il sistema andrebbe arricchito con la capacità di gestire scenari complessi, 
come il cambio di corsia del veicolo \emph{leader} o l'intromissione improvvisa di un altro veicolo nella propria corsia.  
Questi casi, frequenti nella guida reale, richiedono un controllo più flessibile e reattivo, 
magari combinando la logica fuzzy con modelli predittivi o approcci basati sul machine learning.
\\\\
\noindent Un ulteriore passo in avanti sarebbe l'estensione del modello a contesti non autostradali, 
come strade urbane ed extraurbane, dove i limiti di velocità, le interazioni con pedoni e ciclisti, 
e la maggiore variabilità del traffico rendono necessarie regole di controllo più articolate. 
\\\\ 
\noindent Sarebbe inoltre opportuno introdurre un limite di velocità massimo dinamico, 
basato sul riconoscimento automatico dei cartelli stradali o sull'integrazione con i sistemi di navigazione 
che forniscono i limiti della strada che si sta percorrendo.
\\\\
\noindent Un aspetto rilevante da approfondire riguarda l'estensione degli intervalli accettabili per le variabili linguistiche, 
così da ampliare i casi in cui il sistema può operare correttamente.  
In parallelo, sarebbe auspicabile sviluppare una soluzione più elegante rispetto all'attuale strategia, 
che approssima un valore fuori range al limite valido più vicino.
\\\\
\noindent Un ulteriore punto aperto riguarda la definizione del valore della variabile meteorologica, 
attualmente espresso su un intervallo continuo tra 0 e 1.  
Sarebbe necessario stabilire un criterio chiaro, o un algoritmo dedicato, che consenta alla macchina di 
determinare automaticamente il valore corretto in base alle condizioni effettive rilevate 
(ad esempio pioggia leggera, nebbia o forte temporale).
\\\\
\noindent Infine, ulteriori sviluppi potrebbero includere:
\begin{itemize}
    \item l'integrazione con sistemi di comunicazione Vehicle-to-Vehicle (V2V) e Vehicle-to-Infrastructure (V2I), 
    per migliorare la reattività del sistema attraverso lo scambio di informazioni in tempo reale;
    \item l'ottimizzazione del comfort di guida tramite l'adattamento dinamico dei parametri fuzzy 
    in base allo stile di guida del conducente;
    \item la definizione di meccanismi di \textit{self-tuning}, in grado di aggiornare automaticamente 
    le membership functions e le regole in funzione dei dati raccolti durante l'utilizzo reale.
\end{itemize}

\noindent In conclusione, il sistema sviluppato fornisce una base solida ma aperta a numerosi miglioramenti.  
La sua evoluzione futura dovrà mirare a incrementare non solo l'accuratezza e la sicurezza, 
ma anche la capacità di adattarsi a scenari complessi e diversificati, avvicinandosi sempre di più 
a un comportamento di guida realmente autonomo e affidabile.  
